\begin{appendices}
\chapter*{\textsc{Annexe A}}
	\addcontentsline{toc}{chapter}{\textsc{Annexe A}}		
	
	Code MATLAB pour les moindres carrés basiques ,\label{Annexe A} \hyperref[section 1.2]{cliquer pour retourner vers section 1.2}.

	\begin{lstlisting}	
	 
clc
clear all
close all

load donnees.mat

M=417; % instant de commutation
k=1;
u=1;
o=1;
ov=1;
e=1;
p=1;
f=1;
d=1;
oo=1;
dd=1;
ii=1;
ddd=1;
ooo=1;
iii=1;
H=507;

% MC basiques de la droite 1 
for i=1:M+1
A1(o,:)= [ZT(i,2) 1]; % phi = A
y1(i)=ZT(i,1);
o= o+1;
end

% MC basiques pour la droite 2
for j=M+1:H
A2(e,:)= [ZT(j,2) 1];
y2(f)=ZT(j,1);
e= e+1;
f=f+1;
end

C1=inv(A1'*A1)*A1'*y'; % P1 chapeau
C2=inv(A2'*A2)*A2'*y'; % P2 chapeau

for i=1:M+1
ZT1(i)= C1(1)*ZT(i,2)+C1(2);
end

for i=M+1:H
ZT2(d)= C2(1)*ZT(i,2)+C2(2);
d=d+1;
end

plot(ZT(1:M+1,2),ZT1') % plot de la 1ere droite
hold on
plot(ZT(M+1:H,2),ZT2') % plot de la 2eme droite
hold on
plot (ZT(:,2),ZT(:,1),'g') % plot des donnees
hold on

	\end{lstlisting}
	
	\chapter*{\textsc{Annexe B}}
	\addcontentsline{toc}{chapter}{\textsc{Annexe B}}		
	
	\label{Annexe B} \hyperref[section 1.3.1]{ Cliquer pour retourner vers section 1.3.1}.

	\begin{lstlisting}	
	 
for i=1:3 %%  Prise des trois premieres mesures pour la 1ere courbe 
phiz(oo,:)= [ZT(i,2) 1]; % calcul de phi_zero de la droite 1
yr1(oo)=ZT(i,1);
oo= oo+1;
end

Az1= phiz'*phiz; % A_zero de la droite 1
az1=Az1; % pour la question 3.c

 % Calcul du P01 chapeau initial a laide des MC basiques
 Cr1=inv(phiz'*phiz)*phiz'*yr1';
cr1=Cr1; % pour la question 3.c
invAz1 = inv (Az1);
invaz1=invAz1; % pour la question 3.c

 %Prise des trois premieres mesures apres l'instant de commutation pour la 2eme courbe 
for i=M:M+2
phiz2(ooo,:)= [ZT(i,2) 1]; % calcul de phi_zero de la droite 2
yr2(ooo)=ZT(i,1);
ooo= ooo+1;
end

Az2= phiz'*phiz2; % A_zero de la droite 2
Cr2=inv(phiz2'*phiz2)*phiz2'*yr2';% P02 chapeau
invAz2 = inv (Az2);

for i=1:M+1 % application de MC recursifs pour la 1ere droite
yr1mod=[ZT(i,2) 1]*Cr1;
Kn1plus = (invAz1*[ZT(i,2) 1]')/(1+[ZT(i,2) 1]*(invAz1*[ZT(i,2) 1]')); %KN+1
invAz1 = invAz1 - Kn1plus*[ZT(i,2) 1]*invAz1; % l'inverse de AN+1
Cr1= Cr1 + Kn1plus*(ZT(i,1) - yr1mod); %PN+1 chapeau 
end

for i=M+1:H % application de MC recursifs pour la 2eme droite
yr2mod=[ZT(i,2) 1]*Cr2;
YR2MOD(iii)=yr2mod;
Kn2plus = (invAz2*[ZT(i,2) 1]')/(1+[ZT(i,2) 1]*(invAz2*[ZT(i,2) 1]'));
invAz2 = invAz2 - Kn2plus*[ZT(i,2) 1]*invAz2;
Cr2= Cr2 + Kn2plus*(ZT(i,1) - yr2mod);
end

for i=1:M+1
ZTr1(dd)= Cr1(1)*ZT(i,2)+Cr1(2);
dd=dd+1;
end

for i=M+1:H
ZTr2(ddd)= Cr2(1)*ZT(i,2)+Cr2(2);
ddd=ddd+1;
end

plot(ZT(1:M+1,2),ZTr1,'r')
hold on
plot(ZT(M+1:H,2),ZTr2,'r')
grid on
	 
	\end{lstlisting}

\chapter*{\textsc{Annexe C}}
	\addcontentsline{toc}{chapter}{\textsc{Annexe C}}
	
	\label{Annexe C} \hyperref[section 1.3.2] {Cliquer pour retourner vers section 1.3.2 }.
	
	\begin{lstlisting}	
clc
clear all
close all
load donnees.mat
% Question 2
M=417;
k=1;
u=1;
o=1;
ov=1;
e=1;
p=1;
f=1;
d=1;
oo=1;
dd=1;
ii=1;
ddd=1;
ooo=1;
iii=1;
H=507;	

for i=1:3 %% 1ere courbe
phiz(oo,:)= [ZT(i,2) 1];
yr1(oo)=ZT(i,1);
oo= oo+1;
end
Az1= phiz'*phiz;
az1=Az1; % pour la question 3.c
Cr1=inv(phiz'*phiz)*phiz'*yr1';%% P01 chapeau
cr1=Cr1; % pour la question 3.c
invAz1 = inv (Az1);
invaz1=invAz1; % pour la question 3.c
 
for i=1:H
Y1=[ZT(i,1)];
Y1mod=[ZT(i,2) 1]*cr1;
Kn1p = (invaz1*[ZT(i,2) 1]')/(1+[ZT(i,2) 1]*(invaz1*[ZT(i,2) 1]'));
invaz1 = invaz1 - Kn1p*[ZT(i,2) 1]*invaz1;
cr1= cr1 + Kn1p*(ZT(i,1) - Y1mod);
err(i) = abs(Y1-Y1mod); % calcul de l'erreur de prediction pour chaque iteration
if(err(i)>4) 	
 	v=i;
 	break;
end
end
		
for i=1:v-2
zt1(i)= cr1(1)*ZT(i,2)+cr1(2);
end	

plot(ZT(1:v-2,2),zt1,'m')
hold on
plot (ZT(:,2),ZT(:,1),'g')
grid on		
	\end{lstlisting}
	
	\chapter*{\textsc{Annexe D}}
	\addcontentsline{toc}{chapter}{\textsc{Annexe D}}
	
	\label{Annexe D} \hyperref[section 1.3.3] {Cliquer pour retourner vers section 1.3.3 }.
	
	\begin{lstlisting}
clc
clear all
close all
load donnees.mat

M=417;
k=1;
u=1;
o=1;
ov=1;
e=1;
p=1;
f=1;
d=1;
oo=1;
dd=1;
ii=1;
ddd=1;
ooo=1;
iii=1;
H=507;

for i=1:3 %% 1ere courbe
phiz(oo,:)= [ZT(i,2) 1];
yr1(oo)=ZT(i,1);
oo= oo+1;
end
Az1= phiz'*phiz;
az1=Az1; % pour la question 3.c
Cr1=inv(phiz'*phiz)*phiz'*yr1';%% P01 chapeau
cr1=Cr1; % pour la question 3.c
invAz1 = inv (Az1);
invaz1=invAz1; % pour la question 3.c
	
i = 0;
j=1;
Niter=88;
result=0;
error=0;
for i=1:507
Y1=[ZT(i,1)];
Y1mod=[ZT(i,2) 1]*cr1;
Kn1p = (invaz1*[ZT(i,2) 1]'/(1+[ZT(i,2) 1]*(invaz1*[ZT(i,2) 1]'));
invaz1 = invaz1 - Kn1p*[ZT(i,2) 1]*invaz1;
cr1= cr1 + Kn1p*(ZT(i,1) - Y1mod);
err= abs(Y1-Y1mod);

if (err>4)
v=i;

for(j=v:H-Niter)
for (k=v:v+Niter-1)

error=err;
if(error>4)
result=result+1;
end
if (result==Niter)
for s=1:3
fiz(ov,:)= [ZT(s+i,2) 1];
Yr2(ov)=ZT(s+i,1);
ov= ov+1;
end
az2= fiz'*fiz;
cr2=inv(fiz'*fiz)*fiz'*Yr2';
invaz2 = inv (az2);
for u=1:507-i
Y2mod=[ZT(u+i,2) 1]*cr2;
Kn2p = (invaz2*[ZT(i+u,2) 1]')/(1+[ZT(i+u,2)
1]*(invaz2*[ZT(i+u,2) 1]'));
invaz2 = invaz2 - Kn2p*[ZT(i+u,2) 1]*invaz2;
cr2= cr2 + Kn2p*(ZT(i+u,1) - Y2mod);
end
end
end
v=v+1;
end
break
end
end

for i=1:v-2
zt1(i)= cr1(1)*ZT(i,2)+cr1(2);
end
sd=1;
for i=v-2:H
zt2(sd)= cr2(1)*ZT(i,2)+cr2(2);
sd=sd+1;
end

plot(ZT(1:v-2,2),zt1,'m')
hold on
plot(ZT(v-2:H,2),zt2,'m')
hold on
plot (ZT(:,2),ZT(:,1),'g')
grid on						
	\end{lstlisting}
		
	\chapter*{\textsc{Annexe E}}
	\addcontentsline{toc}{chapter}{\textsc{Annexe E}}
	
	\label{Annexe E} \hyperref[section 1.3.4] {Cliquer pour retourner vers section 1.3.4 }.
	
	\begin{lstlisting}	
clc
clear all
close all
load donnees.mat

M=417;
k=1;
u=1;
o=1;
ov=1;
e=1;
p=1;
f=1;
d=1;
oo=1;
dd=1;
ii=1;
ddd=1;
ooo=1;
iii=1;
H=507;

A0=15000*eye(2); % choix arbitraire de la matrice A0
a0=A0;
P0= [12;5];% choix arbitraire des parametres
p0=[150;200]; 
i = 0;
j=1;
Niter=88;
result=0;
error=0;

for i=1:507
Y1=[ZT(i,1)];
Y1mod=[ZT(i,2) 1]*P0;
Kn1p = (A0*[ZT(i,2) 1]')/(1+[ZT(i,2) 1]*(A0*[ZT(i,2) 1]'));
A0 =A0 - Kn1p*[ZT(i,2) 1]*A0;
P0= P0 + Kn1p*(ZT(i,1) - Y1mod);
err= abs(Y1-Y1mod);


if (err>4)
v=i;

for(j=v:H-Niter)
for (k=v:v+Niter-1)
error=err;
if(error>4)
result=result+1;
end
if (result==Niter)

A01=10000*eye(2);
P01= [5;18];
for u=1:507-i
Y2mod=[ZT(u+i,2) 1]*P01;
Kn2p = (A01*[ZT(i+u,2) 1]')/(1+[ZT(i+u,2)
1]*(A01*[ZT(i+u,2) 1]'));
A01 = A01 - Kn2p*[ZT(i+u,2) 1]*A01;
P01= P01 + Kn2p*(ZT(i+u,1) - Y2mod);
end
end
end
v=v+1;
end
break
end
end

for i=1:v-2
zt1(i)= A0(1)*ZT(i,2)+A0(2);
end

for i=v:507
Y1=[ZT(i,1)];
Y1mod=[ZT(i,2) 1]*p0;
Kn1p = (a0*[ZT(i,2) 1]')/(1+[ZT(i,2) 1]*(a0*[ZT(i,2) 1]'));
a0 =a0 - Kn1p*[ZT(i,2) 1]*a0;
p0= p0 + Kn1p*(ZT(i,1) - Y1mod);
end
sd=1;
for i=v+2:H
zt2(sd)= a0(1)*ZT(i,2)+a0(2);
sd=sd+1;
end
plot(ZT(1:v-2,2),zt1,'m')

hold on
plot(ZT(v+2:H,2),zt2,'m')
hold on
plot (ZT(:,2),ZT(:,1),'g')
grid on				
	\end{lstlisting}
	
\end{appendices}
