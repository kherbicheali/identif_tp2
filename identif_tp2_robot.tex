\documentclass[12pt, a4paper, openany]{report}
\def\VersionRapport{1.0}
\usepackage[utf8]{inputenc} % un package
\usepackage[T1]{fontenc}      % un second package
\usepackage[francais]{babel}  % un troisième package
\usepackage{layout}
\usepackage[top=2.7cm, bottom=2.5cm, left=3.5cm, right=3cm]{geometry}
\usepackage{setspace}
\frenchbsetup{StandardLists=true} % à inclure si on utilise \usepackage[french]{babel}
%\usepackage{enumitem}
\usepackage[shortlabels]{enumitem}
\usepackage{amssymb}
\usepackage{color}
\usepackage{listings}
\definecolor{dkgreen}{rgb}{0,0.6,0}
\definecolor{gray}{rgb}{0.5,0.5,0.5}
\definecolor{mauve}{rgb}{0.58,0,0.82}
\definecolor{rougecerise}{rgb}{0.73,0.043,0.043}
\lstset{frame=tb,
  language=Matlab,
  aboveskip=3mm,
  belowskip=3mm,
  showstringspaces=false,
  columns=flexible,
  basicstyle={\small\ttfamily},
  keywordstyle=\color{blue},
  commentstyle=\color{dkgreen},
  stringstyle=\color{mauve},
  breaklines=true,
  breakatwhitespace=true,
  tabsize=3,
  breaklines=true,
  morekeywords={matlab2tikz},
  morekeywords=[2]{1}, 
  keywordstyle=[2]{\color{black}},
  identifierstyle=\color{black},
  numbers=left,
  numberstyle={\tiny \color{black}},
  numbersep=9pt, 
  emph=[1]{for,end,break},
  emphstyle=[1]\color{red}
}
\usepackage{multirow} % pour les tableaux
\usepackage[table]{xcolor} % pour les tableaux
\usepackage{verbatim}
%\usepackage{subcaption}
\usepackage{graphicx}
\usepackage{moreverb}
\usepackage{url}
\usepackage{pst-all}
\usepackage{eso-pic,graphicx}
\usepackage{caption} 
\usepackage[colorlinks=true,urlcolor=blue,linkcolor=red]{hyperref}
\usepackage{array}
\usepackage[toc,page]{appendix}
\usepackage[off]{auto-pst-pdf}
\usepackage{hyperref} % pour le sommaire table des matières
\AddThinSpaceBeforeFootnotes % à insérer si on utilise \usepackage[french]{babel}
\FrenchFootnotes % à insérer si on utilise \usepackage[french]{babel}
\usepackage{fancyhdr}
\pagestyle{headings}
\usepackage{pifont}  %pour les puces
\usepackage{amsmath} %pour les puces
\usepackage{subfig}
\usepackage{verbatim} % pour le code en annexe 
%%%%%%%colones 
\newcolumntype{R}[1]{>{\raggedleft\arraybackslash }b{#1}}
\newcolumntype{L}[1]{>{\raggedright\arraybackslash }b{#1}}
\newcolumntype{C}[1]{>{\centering\arraybackslash }b{#1}}
%%%%%%% 
\renewcommand{\appendixpagename}{Annexes}
\renewcommand{\appendixtocname}{Annexes}
\title{Theme: Compte Rendu Systèmes linéaires à temps discret et identification}
\author{NAIMI \bsc{Nabil} \\ KHERBICHE \bsc{Ali}}
\date{2018-2019}
%new
\newcommand{\HRule}{\rule{\linewidth}{0.5mm}}

\begin{document}

%\selectlanguage{francais}
\pagenumbering{arabic} 
\makeatletter
\begin{titlepage}
\begin{sffamily}
\begin{center}
    % Upper part of the page. The '~' is needed because \\
    % only works if a paragraph has started.
    \includegraphics[scale=0.5]{Logo_UT3.jpg}~\\[1cm]
    \textsc{\LARGE M1 ISTR-RODECO  }\\[1cm]
    \textsc{\Large Compte Rendu Systèmes Linéaires à Temps Discret et Identification}\\[1cm]
    % Title
    \HRule \\[0.4cm] % saut de ligne
    { \huge  \textsc {Reconstruction de l'environnement d'un robot mobile à l'aide de la méthode des moindres carrés \\[0.4cm] }}

    \HRule \\[1cm]   % sous de ligne 
    \includegraphics[scale=0.1]{logomaster.jpg}
    \\[1cm]
    % Author and supervisor
    \begin{minipage}{0.4\textwidth}
      \begin{flushleft} \large
         \textsc{\emph {Fait par:} \\KHERBICHE Ali\\ NAIMI Nabil}  
          \newline
          Promotion 2018-2019 \\
      \end{flushleft}
    \end{minipage}
    \begin{minipage}{0.4\textwidth}
      \begin{flushright} \large
        %%\emph{Tuteur et}
        \emph{Tuteurs:} \textsc{Mme.Carine JAUBERTHIE\\Mme.Viviane CADENAT}
      \end{flushright}
    \end{minipage}
    \vfill
    % Bottom of the page
    {\large Mars 2018}
  \end{center}
  \end{sffamily}                
  \end{titlepage}  
\makeatother
   
%*********************** somaire **************
\renewcommand{\contentsname}{Sommaire}
\tableofcontents
%*********************** listes des figures **************
\listoffigures
%*********************** listes des tableaux **************
%\listoftables

\chapter*{\textsc{Introduction}}
\addcontentsline{toc}{chapter}{\textsc{Introduction}}

	\paragraph{}
	L'objectif de cette séance est d'étudier le problème de la localisation d'un robot mobile (Pekee II). Nous traiterons ce problème par la méthode des moindres carrés récursifs.
	
	\begin{center}
	\includegraphics[scale=0.5]{pekee.png}
	\captionof{figure}{\textit{Robot pekee II\\}}
	\label{fig1} 
	\end{center}   
	
\chapter{\textsc{Reconstruction de l'environnement}}
%\addcontentsline{toc}{chapter}{\textsc{Reconstruction de l'environnement}}

\section{\textsc{Tracé des données}}

	\paragraph{}
	D'après le graphe ci-dessous qui provient du fichier ZT, l'axe des ordonnées correspond à la première colonne 			qui est en fonction de la deuxième celle ci correspond aux abscisses. On constate que le laser fournit un 				ensemble de points sous forme de deux pseudo-droites qu'on pourra modéliser en droites par la méthode des 				moindres carrés.\\  
	
	\begin{center}
	\includegraphics[scale=0.5]{graphedonnees.png}
	\captionof{figure}{\textit{Tracé des données du robot\\}}
	\label{fig2} 
	\end{center} 

\pagebreak
\section{\textsc{Identification par les moindres carrés basiques}}	

	\paragraph{}
	On sait que l'équation d'une droite est de la forme $ y=ax+b $ qu'on pourra illustrer aussi de cette manière :\\
	\begin{center}
		$ y = \begin{bmatrix} x&1 \end{bmatrix} \begin{bmatrix} a\\b \end{bmatrix} $ 
	\end{center}
	
	Arrivés là une identification avec la forme $ y = \varphi^{T}P $ est possible avec :
	  
	\begin{center}
		$ \varphi^{T} = \begin{bmatrix} x&1 \end{bmatrix} $ et $ P = \begin{bmatrix} a\\b \end{bmatrix}$
	\end{center}
	Ou $P$ correspond aux paramètres et $\varphi^{T}$ correspond au vecteur de régression. Le fichier ZT contient 			deux colonnes: colonne 1 pour les Y et colonne 2 	pour les X, donc on aura exactement N=507 mesures alors on 			parle de matrice de régression $\phi$ et pas que de vecteur de régression $\varphi$, avec $ \phi = 					\begin{bmatrix} \varphi^{T}_{1}\\.\\ \varphi^{T}_{i} \\.\\ \varphi^{T}_{N} \end{bmatrix} $ et $ Y = 					\begin{bmatrix} y_{1}\\.\\ y_{i} \\.\\ y_{N} \end{bmatrix} $ ,\label{section 1.2} \hyperref[Annexe A] {voir code 	Matlab en annexe}.
	
	\begin{center}
	\includegraphics[scale=0.7]{2.jpg}
	\captionof{figure}{\textit{Tracé des droites modélisées par les MC basiques superposé avec celui des données\\}}
	\label{fig3} 
	\end{center}

	\textbf{Nota:} La détection de l'instant de commutation à i=417 nous a permis de tracer les deux modéles. \\[1cm] 
	
	\par Cette approche est limitée car elle traite un mur à la fois, autrement dit un modèle par mur. Mais comme la problématique abordée comporte plus d'un mur l'algorithme de cette approche risque d'être très long et quasiment impossible à coder pour une situation réelle ou on affronte une multitude d'obstacles (murs...etc). 
	

\section{\textsc{Identification par les moindres carrés récursifs}}
\subsection{\textsc{ Principe et erreur de prédiction }}

	\paragraph{} L'algorithme des moindres carrés récursifs se compose de deux parties, partie initialisation et partie calcul itératif des parmaètres. Il existe deux solutions pour initialiser cet algorithme, soit avec un choix arbitraire des paramètres ou bien de faire un petit prélévement des mesures donc celles du fichier ZT. Dans notre cas on se penchera plus tôt sur la deuxième option c'est à dire on prélève un nombre de mesures $N_0$ qui est égale dans notre cas à $3$ puis en utilisant les moindres carrés basiques on poura facilement trouver la matrice de régression $\phi_0$ ainsi que les paramètres $P_0$ et la matrice $A_0$ qui est égale à $\phi^{T}_{0} \phi_0$.\\
Le calcul itératif des paramètres consiste à acquérir à chaque itération une nouvelle mesure $y_{N+1}$, de calculer son $y_{MOD_{N+1}} = \varphi^{T}_{N+1} \hat{P}_N$, ainsi que la matrice récursive  $A^{-1}_{N+1}$ avec la relation : $A^{-1}_{N+1} = A^{-1}_{N} - K_{N+1}\varphi^{T}_{N+1} A^{-1}_{N} $ ou $K_{N+1} = \frac{ A^{-1}_{N}\varphi_{N+1} }{ 1+\varphi^{T}_{N+1} A^{-1}_{N} \varphi_{N+1}  }$. Pour arriver enfin à exprimer $\hat{P}_{N+1} $ en fonction de $\hat{P}_N$ comme suit :  $\hat{P}_{N+1} = \hat{P}_N + K_{N+1} (y_{N+1} - y_{MOD_{N+1}}) $
, \label{section 1.3.1} \hyperref[Annexe B] {voir code Matlab en annexe}. \\

	\par L'erreur de prédiction équivaut à $ y_{N+1} - y_{MOD_{N+1}} $ qui évolue à chaque itération, quand cette dernière augmente en valeur ça explique que l'écart entre l'evolution réelle $y_{N+1}$ et l'évolution modélisée $y_{MOD_{N+1}}$ augmente, lorsque l'erreur de prédiction arrive à un seuil donné il vaut mieux prévoir de stopper l'estimation et de faire une seconde modélisation afin d'être toujours proche de la réalité car on fait face à ce qu'on appelle l'instant de commutation.\\    

	\begin{center}
	\includegraphics[scale=0.5]{MCrecu.PNG}
	\captionof{figure}{\textit{Tracé des droites modélisées par les MC récursifs superposé avec celui des données\\}}
	\label{fig4} 
	\end{center}
	
	\textbf{Nota:} La figure ci-dessus est tracée en prenant en compte l'instant de commutation trouvé précédemment c'est-à-dire à $i=417$.\\ 
	
\subsection{\textsc{ Reconstruction d'un mur avec utilisation de l'erreur de prédiction }}

	\paragraph{} On suppose que le nombre de murs présents est \textbf{2}, on fixe le sueil de l'erreur de prédiction à \textbf{4}. Il suffit de prendre le code précédent et de lui ajouter une condition d'arrêt, \label{section 1.3.2} \hyperref[Annexe C] {voir code Matlab en annexe}. \\
  
  \begin{center}
	\includegraphics[scale=0.5]{3_b.PNG}
	\captionof{figure}{\textit{Tracé d'une droite modélisée par les MC récursifs avec utilisation de l'erreur de prédiction superposé avec celui des données\\}}
	\label{fig5} 
	\end{center}
	
\subsection{\textsc{ Reconstruction d'un mur avec utilisation de l'erreur de prédiction pendant un nombre d'itérations NITER}}
  
  \paragraph{} Afin d'éviter toutes les anomalies qui peuvent bruiter ou perturber les mesures, la prolongation de l'erreur de prédiction sur un nombre NITER d'itérations semble être une bonne solution.\\
  Après plusieurs tests $NITER = 88$ apporte la meilleure approche possible, \label{section 1.3.3} \hyperref[Annexe D] {voir code Matlab en annexe}. \\
  
	\begin{center}
	\includegraphics[scale=0.5]{NITER.PNG}
	\captionof{figure}{\textit{Tracé des droites modélisées par les MC récursifs avec utilisation de l'erreur de prédiction pendant NITER itérations superposé avec celui des données\\}}
	\label{fig6} 
	\end{center}
   
   \par \textbf{Conclusion :} Avec cette méthode, la détection de l'instant de commutation se fera automatiquement pour un nombre de murs connus par avance mais qui est impossible pour $n$ murs inconnus.\\
      
\subsection{\textsc{ Reconstruction des murs avec initialisation arbitraire de l'algorithme}}

	\paragraph{} On initialise d'abord les paramètres $P_0$ arbitrairement, de même pour la matrice $A_0$ qui doit s'écrire comme suit : $A_0 = C*Id$, avec $C$ une constante de grande valeur $10^{6}$ dans notre cas et $Id$ une matrice identité $2$x$2$ dans notre cas puisqu'on a $2$ paramètres, \label{section 1.3.4} \hyperref[Annexe E] {voir code Matlab en annexe}. \\
	
		\begin{center}
	\includegraphics[scale=0.5]{E.PNG}
	\captionof{figure}{\textit{Tracé des droites modélisées par les MC récursifs avec initialisation arbitraire des paramètres superposé avec celui des données\\}}
	\label{fig7} 
	\end{center}
	   
	\textbf{Nota:} Cette méthode est moins précise que sa précédente vu qu'on initialise arbitrairement $A_0$.	   
	   
\subsection{\textsc{ Conclusion }}

	On pense que l'algorithme précédent doit être mis sous forme d'une fonction qui sera appelée à chaque détection d'un instant de commutation.
	  
%\input{chap2.tex}
%\input{chap4&5.tex}
%\input{conclusion.tex}
\begin{appendices}
\chapter*{\textsc{Annexe A}}
	\addcontentsline{toc}{chapter}{\textsc{Annexe A}}		
	
	Code MATLAB pour les moindres carrés basiques ,\label{Annexe A} \hyperref[section 1.2]{retour vers section 1.2}.

	\begin{lstlisting}	
	 
clc
clear all
close all

load donnees.mat

M=417; % instant de commutation
k=1;
u=1;
o=1;
ov=1;
e=1;
p=1;
f=1;
d=1;
oo=1;
dd=1;
ii=1;
ddd=1;
ooo=1;
iii=1;
H=507;

% MC basiques de la droite 1 
for i=1:M+1
A1(o,:)= [ZT(i,2) 1]; % phi = A
y1(i)=ZT(i,1);
o= o+1;
end

% MC basiques pour la droite 2
for j=M+1:H
A2(e,:)= [ZT(j,2) 1];
y2(f)=ZT(j,1);
e= e+1;
f=f+1;
end

C1=inv(A1'*A1)*A1'*y'; % P1 chapeau
C2=inv(A2'*A2)*A2'*y'; % P2 chapeau

for i=1:M+1
ZT1(i)= C1(1)*ZT(i,2)+C1(2);
end

for i=M+1:H
ZT2(d)= C2(1)*ZT(i,2)+C2(2);
d=d+1;
end

plot(ZT(1:M+1,2),ZT1') % plot de la 1ere droite
hold on
plot(ZT(M+1:H,2),ZT2') % plot de la 2eme droite
hold on
plot (ZT(:,2),ZT(:,1),'g') % plot des donnees
hold on

	\end{lstlisting}
	
	\chapter*{\textsc{Annexe B}}
	\addcontentsline{toc}{chapter}{\textsc{Annexe B}}		
	
	\label{Annexe B} \hyperref[section 1.3.1]{retour vers section 1.3.1}.

	\begin{lstlisting}	
	 
for i=1:3 %% 1ere courbe
phiz(oo,:)= [ZT(i,2) 1]; % calcul de phi_zero de la droite 1
yr1(oo)=ZT(i,1);
oo= oo+1;
end

Az1= phiz'*phiz; % A_zero de la droite 1
az1=Az1; % pour la question 3.c
Cr1=inv(phiz'*phiz)*phiz'*yr1'; % P01 chapeau
cr1=Cr1; % pour la question 3.c
invAz1 = inv (Az1);
invaz1=invAz1; % pour la question 3.c

for i=M:M+2
phiz2(ooo,:)= [ZT(i,2) 1]; % calcul de phi_zero de la droite 2
yr2(ooo)=ZT(i,1);
ooo= ooo+1;
end

Az2= phiz'*phiz2; % A_zero de la droite 2
Cr2=inv(phiz2'*phiz2)*phiz2'*yr2';% P02 chapeau
invAz2 = inv (Az2);

for i=1:M+1 % application de MC recursifs pour la 1ere droite
yr1mod=[ZT(i,2) 1]*Cr1;
Kn1plus = (invAz1*[ZT(i,2) 1]')/(1+[ZT(i,2) 1]*(invAz1*[ZT(i,2) 1]'));
invAz1 = invAz1 - Kn1plus*[ZT(i,2) 1]*invAz1;
Cr1= Cr1 + Kn1plus*(ZT(i,1) - yr1mod);
end

for i=M+1:H % application de MC recursifs pour la 2eme droite
yr2mod=[ZT(i,2) 1]*Cr2;
YR2MOD(iii)=yr2mod;
Kn2plus = (invAz2*[ZT(i,2) 1]')/(1+[ZT(i,2) 1]*(invAz2*[ZT(i,2) 1]'));
invAz2 = invAz2 - Kn2plus*[ZT(i,2) 1]*invAz2;
Cr2= Cr2 + Kn2plus*(ZT(i,1) - yr2mod);
end

for i=1:M+1
ZTr1(dd)= Cr1(1)*ZT(i,2)+Cr1(2);
dd=dd+1;
end

for i=M+1:H
ZTr2(ddd)= Cr2(1)*ZT(i,2)+Cr2(2);
ddd=ddd+1;
end

plot(ZT(1:M+1,2),ZTr1,'r')
hold on
plot(ZT(M+1:H,2),ZTr2,'r')
grid on
	 
	\end{lstlisting}
			
\end{appendices}
%*********************** Bibliographie ************ 
\bibliographystyle{alpha}
\bibliography{biblio}  



\end{document}
